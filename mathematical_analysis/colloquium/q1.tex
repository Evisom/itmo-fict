\documentclass[14pt]{article}
\usepackage{hyperref}
\usepackage[utf8x]{inputenc}
\usepackage[english,russian]{babel}
\usepackage{cmap}
\usepackage{amsmath}
\usepackage[left=2cm,right=2cm,
    top=2cm,bottom=2cm,bindingoffset=0cm]{geometry}

\title{Билеты по математическому анализу для коллоквиума 14 ноября }
\author{Шишминцев Дмитрий Владимирович}

\begin{document}
    \maketitle
    \tableofcontents
    \newpage
    
    \section{Множества и операции над ними}
        \textsc{(Условно) Определение: }
        Множество - совокупность некоторых объектов определенных по одному признаку. \\
        $ a \in A $ - элемент a принадлежит множеству A \\ 
        $ a \notin A $ - элемент a не принадлежит множеству A \\ 
        $ A \subset B $ - множество A является подмножеством B \\
        \\
        \textsc{Равенство множеств: } Множества равны если каждый элемент множества A является элементом множества B и наоборот \\
        $A = B \Leftrightarrow \begin{cases}
            x \in A \Rightarrow x \in B \\ 
            x \in B \Rightarrow x \in A
        \end{cases}$\\
        \\
        \textsc{Операции над множествами:}
        \begin{itemize}
            \item Пересечение множеств: $ A \cup B $
            \item Объединение множеств: $ A \cap B $ 
            \item Разность множеств: $ A \backslash B $
            \item Симметричная разность: $ A \bigtriangleup B = (A \backslash B) \cap (B \backslash A)$
            \item Декартово произведение множеств: $ A \times B = \{(a; b) | a \in A, b \in B \}$
        \end{itemize}



\end{document}