\chapter{Лекция 9}
\subsubsection{Замена переменных в двойном интеграле}
$J$ - якобиан, коэффицент растяжения 
$J = \lim_{diam S^k \rightarrow 0} \frac{\Delta S}{\Delta S^k}$

Пусть есть непрерывная функция $x = \phi(u, v); y = \xi (u,v)$ и однозначно отображают $D $ в $D^*$ и эти функции имеют непрерывную частную производную. 
Пусть в области $D$ на плоскости $Oxy$ задана функция $z = f(x,y)$ и ей соответстует функция $f(\phi(u,v), \xi(u,v))$. Тогда сумма Римана $\sum_D f(x,y) \Delta S = \sum_D f(\phi(u,v), \xi(u,v)) \cdot J \Delta S $
$$|J| = \begin{bmatrix}
    x_u' & x_v' \\ 
    y_u' & y_v'
\end{bmatrix}$$
Если $\lim_{\alpha \rightarrow 0} \sum_D f(x,y)\Delta S = \iint_D f(x,y) dxdy$ - существует конечное значение 
$$
\iint_D f(x,y) dxdy = \iint_{D^*} f(\phi(u,v), \xi(u,v)) \cdot |J| dudv
$$

\subsubsection{Двойной интеграл в полярных координатах}

$\\ x(r, \phi) = x = r\cdot \cos \phi \\ y(r, \phi) = y = r \cdot \sin \phi \\ r \in [0, +\infty]; \phi \in (0, 2\pi] \Rightarrow $

$$ |J| = 
\begin{bmatrix}
    x_r' & x_\phi' \\ 
    y_r' & y_\phi'
\end{bmatrix} = \begin{bmatrix}
    \cos \phi & r\cdot(-\sin \phi) \\ 
    \sin \phi & r\cdot \cos \phi
\end{bmatrix} = |r\cdot \cos^2 \phi + r \cdot \sin^2 \phi| 
$$

$$
\iint_D f(x,y) dxdy = \iint_{D^*} f(r\cos \phi, r\cos\phi) \cdot r drdy
$$

\subsubsection{Тройной интеграл}
\begin{definition}[Тройной интеграл]
    Дано материальное тело, представляющее собой пространственную область $\Omega$ заполненную массой. Требуется найти массу этой области, при условии что в каждой точке этой области известна плотность. $\phi(P) = \phi(x,y,z)$
    Разобъем $\Omega$ на неперекрывающиеся кубируемые части: $\Omega_1, \Omega_2, \Omega_3, ... , \Omega_n$ в соотсветствии с объемами $\Delta v_1, \Delta v_2, ... , \Delta v_n$. В каждой области выбираем $\forall (\cdot) P_k $ с плотностью $\phi(p_k)$, тогда масса этой области $\Delta m_k \approx \phi(P_k) \cdot \Delta v_k$, мааса всей области $\Omega m \approx \sum_{k=1}^{n} \phi(P_k) \cdot v_k$. Пусть $d$ наименьший из диаметров частичных областей: $\lim_{d\rightarrow 0} \sum_{k=1}^{n} \phi(P_k) \cdot \Delta v_k$, где сумма не зависит от выбора точки $P_k$, не зависит от разбиения и если предел конечен, то:
    $$
        \iiint_{\Omega} \phi(x,y,z) dxdydz = \lim_{d \rightarrow 0} \sum_{k=1}^n \phi(P_k) \cdot \Delta v_k
    $$
    $$
        \iiint_\Omega f(x,y,z) dxdydz = \iint_D dxdy \int_l^m f(x,y,z ) dz = \int_a^b dx \int^d_c dy \int^m_l f(p) dz 
    $$
\end{definition}


