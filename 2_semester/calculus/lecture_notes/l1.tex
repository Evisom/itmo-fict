% \chapter{Лекция 1}
%     \begin{definition}
%         Функция $F(x)$ называется первообразной функции $f(x)$, если на определенном интервале функция $F(x)$ дифференцируема и удовлетворяет отношению $F'(x) = f(x)$
%     \end{definition}

%     \begin{theorem}
%         $F(x)$ - первообразная для $f(x)$. Тогда $F(x) + C$, где $C$ - произвольная константа содержит в себе все первообразные для $f(x)$
%     \end{theorem}

%     \begin{definition}
%         Совокупность всех первообразных для функции на некотором интервале называется интегралом
%     \end{definition}

%     \begin{theorem}
%         Если функция непрерывна на некотором интервале, то она имеет на этом интервале первообразную, то есть интегрируема
%     \end{theorem}

%     \textsc{Свойства интегралов:}

%     \begin{itemize}
%         \item $(\int f(x) dx)' = f(x)$
%         \item $\int k \cdot f(x) dx = k \cdot \int f(x) dx$ 
%         \item $\int (f_1(x) + f_2(x)) dx = \int f_1(x) dx + \int f_2(x) dx$
%     \end{itemize}