\documentclass[14pt]{extreport}
\usepackage{gost}

\title{Конспект лекций по математическому анализу}
\author{Шишминцев Дмитрий Владимирович}

\begin{document}
    \maketitle 
    \newpage 
    \tableofcontents
    \newpage
    \chapter{Основы анализа}
        \section{Множества}
            \begin{definition}
                Множество - совокупность элементов одной природы и некоторым общим свойством позволяющим объеденить их в одно целое.    
            \end{definition}
            

            \textsc{Обозначения:}
            \begin{itemize}
                \item $A, B, C$ - множества, $a,b,c$ - элементы множества 
                \item $\forall$ - квантор общности (для каждого)
                \item $\exists$ - найдется
                \item $\mathbb{X/E/U}$ - универсальные множества 
                \item $\emptyset$ - пустое множество 
                \item ! - единственность 
                \item $\rightarrow$ - следовательно 
            \end{itemize}

            \textsc{Операции над множествами:}
            \begin{itemize}
                \item $A  \cap B$ - объединение множеств (коммутативно и ассоциативно)
                \item $A \cup B$ - пересечение множеств (коммутативно и ассоциативно)
                \item $ A \backslash B $ - разность множеств 
                \item $\bar{A}$ - отрицание
                \item $A \varDelta B$ - симметрическая разность $(A \cup B) \backslash (A \cap B)$
                \item $A \times B = \{(x,y) | x \in A, y \in B \}$ - декартово произведение
            \end{itemize}

        \section{Отображения (функция)}
        \begin{definition}
            правило по которому $\forall x \in D \exists !y \in V$ 
        \end{definition}
            $F: D$ (область определения) $\rightarrow $ (правило перевода) $V $  (область значений)

            \textsc{Вариативности функциональных отображений:}
            \begin{itemize}
                \item Сюръекция ($\forall y \in V: \exists x \in D$) - каждый элемент в области значений функции имеет прообраз в области определения
                \item Инъекция ($ \forall y \in V : \exists ! x \in D$) - каждый элемент в области определения функции имеет образ в области значений. Не каждый образ имеет прообраз. 
                \item Биекция ($ \forall F : A \rightarrow B \exists ! F^{-1} : B \rightarrow A $ ) - функция яаляется и сюръекцией и биекцией. 
            \end{itemize}
        
        \section{Характеристики множеств}
            \begin{definition}
                Мощность (кардинальное число) -  количество различных элементов множества. 
            \end{definition}

            \begin{definition}
                Эквивалентность множеств: множества эквивалентны $(A \sim B)$ если они равномощны. $\forall x \in X \exists ! y \in Y$ и $\forall y \in Y \exists ! x \in X$
            \end{definition}
                
            \begin{definition}
                Счетность множеств: множество счетно (исчислимо), если $A \sim \mathbb{N}$ 
            \end{definition}

            \begin{definition}
                Мощность континуума: множество эквивалентное множеству точек отрезка $[0,1]$ имеет мощность континуума.
            \end{definition}
            
            \begin{theorem}
                Множество всех точек отрезка $[0;1] $ - несчетно
            \end{theorem}

            \begin{theorem}[Кантора-Бернштейна]
                Если $A \sim B' (B' \subset  B) $ и $B \sim A' (A' \subset A) \Rightarrow A \sim B$\\
                Если $A \subset B \subset C$, причем $A \sim C \Rightarrow A \sim B$
            \end{theorem}

            \begin{definition}[Сравнение мощностей множеств]
                $\exists B' \in B: B' \sim A$ и $\nexists A' \in A: A' \sim B \Rightarrow |A| < |B|$
            \end{definition}
        
        \section{Множества чисел}
        \begin{itemize}
            \item $\mathbb{N}$ - натуральные числа $\{1,2,3...\}$
            \item $\mathbb{Z} $ - целые числа $\{-1, 0, 1, 2..\}$ 
            \item $\mathbb{Q}$ - рациональные числа $\{\frac{2}{3},0.(3)\}$
            \item $\mathbb{R}$ - вещественные (действительные числа) $\{\sqrt{2}, \pi, e \}$
            \item $\mathbb{C} $ - комплексные 
        \end{itemize}
        $\mathbb{N} \subset \mathbb{Z} \subset \mathbb{Q} \subset \mathbb{R} \subset \mathbb{C}$
        
        \textsc{Основные свойства вещественных чисел:}
        \begin{itemize}
            \item Транзитивность ($a>b, b>c \rightarrow a > c$)
            \item Ассоциативность $(a+(b+c) = (a+b)+c)$
            \item Коммутативность $a+b = b+a$
            \item Дистрибутивность $(a+b)\cdot c = a \cdot c + b \cdot c$
            \item $\forall a,b \in \mathbb{R} \exists ! c \in \mathbb{R}: a+b=c$
            \item $\forall a \neq 0 \exists! a^{-1}: a \cdot a^{-1} = 1$
        \end{itemize}
            
        \textsc{Грани множеств:}

        \begin{itemize}
            \item $\forall b \in \mathbb{R}:\forall a \in A \rightarrow a \leq b$ - верхняя грань 
            \item $\forall d \in \mathbb{R}:\forall a \in A \rightarrow d \leq a$ - нижняя грань 
        \end{itemize}

            Грани не единственны

        \begin{definition}
            Точная верхняя/нижняя грань - минимальный/максимальный элемент множества верхних/нижних граней множеств. 
        \end{definition}


        \textsc{Свойство точной верхней грани:}

        Если $ b = \sup A$, то $\forall \epsilon > 0 \exists a \in A: a > b - \epsilon$

        $\blacktriangleright$
            Допустим обратное. Тогда $a \leq b - \epsilon$ А это невозможно т.к b является наименьшей верхней гранью.
        $\blacktriangleleft$
    
        \textsc{Свойство нижней верхней грани:}

        Если $ d = \inf A$, то $\forall \epsilon > 0 \exists a \in A: a < d + \epsilon$

        $\blacktriangleright$
            Док-во аналогично свойству точной верхней грани. 
        $\blacktriangleleft$

        \begin{theorem}[Принцип вложенных отрезков]
            Пусть $\{[a_n, b_n]\}^{\inf}_{n=1}: \forall n \in \mathbb{N} \rightarrow [a_{n+1}, b_{n+1} \subset [a_n, b_n]]$ тогда $\exists ! c \in \mathbb{R}: \forall n \in \mathbb{N} \rightarrow c \in [a_n, b_n]$
        \end{theorem}
        $\blacktriangleright$
            Пусть длина отрезка - $d(n) = b_n - a_n. \forall k \in \mathbb{N} \rightarrow d(1) > d(k)$. Пусть $c := \sup a_n \Rightarrow \forall n \rightarrow a_n \leq c \leq b_n. \forall n \rightarrow c \leq b_n \Rightarrow c \in [a_n, b_n]$. Единственность точки следует из стремления длин отрезков к нулю.
        $\blacktriangleleft$

    \section{Метод математической индукции}
        Для обоснования ММИ используем свойство натуральных чисел: $\forall A \subset \mathbb{N}: A \neq \emptyset \exists a' \in A: \forall a \in A \rightarrow a' \leq a$. Метод математической индукции для док-ва утверждения на множестве $A$ состоит из шагов:
        \begin{itemize}
            \item База индукции - проверяем справедливость на $a'$
            \item Индукционное предположение - проверяем для произвольного элемента $a_k \in A$
            \item Индукционный шаг - доказываем справедливость для $a_{k+1} \in A$
        \end{itemize}

    \section{Бином Ньютона}
        \begin{equation}
            (1+x)^n = \sum^n_{k=0} C^k_n x^k
        \end{equation}
        $C^k_n =\binom{n}{k}= \frac{n!}{k!(n-k)!}$ - биноминальный коэффициент\\
        $\binom{n}{k} + \binom{n}{k+1} = \binom{n+1}{k+1}$

        $\blacktriangleright$
            По методу математической индукции. 
            При $n=1. 1+x = C^0_1 + C^1_1 x = 1+x$
            При $n=t$ формула так же верна. \\ 
            При $n=t+1 \\ (1+x)^{t+1} = (1+x)^t(1+x) = \binom{t}{0}x^0 + ... + \binom{t}{t}x^t + \binom{t}{0} x + ... + \binom{t}{t}x^{t+1} = \binom{t+1}{0} + \binom{t+1}{1}x + ... + \binom{t+1}{t+1} x^{t+1}$
        $\blacktriangleleft$

    \section{Неравенство Бернулли}
        \begin{equation}
            (1+x)^n > 1 + xn
        \end{equation}
        При $x>-1, x\neq 0, n \geq 2$ \\ 
        Док-во по ММИ.


\chapter{Пределы}
        \section{Предел числовой последовательности}
            \begin{definition}[Числовая последовательность]
                
                $\\\beth x_n = f(n), f:\mathbb{N} \rightarrow \mathbb{R}$
            \end{definition}

            Операции с числовыми последовательностями выполняются почленно. 
        
                \begin{definition}[Ограниченность последовательности]
                    $\\ \exists A \in \mathbb{R} : \forall n \in \mathbb{N} \rightarrow |x_n| \leq A$
                \end{definition}


                \begin{definition}[Бесконечно большая последовательность]
                    Последовательность называется бесконечно большой, если множество членов удовлетворяющих условию $|x_n| \leq c$ конечно.
                    $\\ \forall c > 0 \exists n(c) \in \mathbb{N} : \forall n > n(c) \rightarrow |x_n| > c$
                \end{definition}
                
                \begin{definition}[Бесконечно малая последовательность]
                    Последовательность называется бесконечно малой, если множество членов удовлетворяющих условию $|x_n| \geq c$ конечно.
                    $\\ \forall c > 0 \exists n(c) \in \mathbb{N} : \forall n > n(c) \rightarrow |x_n| < c$
                \end{definition}

                \begin{theorem}[Ограниченность бесконечно малой последовательности]
                    Если ${x_n}$ - б.м.п $\Rightarrow \forall n \in \mathbb{N} \rightarrow |x_n| < C, C \in \mathbb{R}_+$
                \end{theorem}
                $\blacktriangleright$
                    По определению бесконечно малой последовательности, кол-во элементов $|x_n| \geq C$ конечно. Возьмем $C = max(|x_1|, |x_2|, ... , |x_n|)$. Получим $\forall n \in \mathbb{N} \rightarrow |x_n| < C$
                $\blacktriangleleft$

        % $\blacktriangleright$ $\blacktriangleleft$
        
\end{document}