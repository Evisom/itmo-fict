\documentclass[14pt]{extreport}
\usepackage{gost}

\title{Конспект лекций по математическому анализу}
\author{Шишминцев Дмитрий Владимирович}

\begin{document}
    \maketitle 
    \newpage 
    \tableofcontents
    \newpage
    \chapter{Основы анализа}
        \section{Множества}
            \begin{definition}
                Множество - совокупность элементов одной природы и некоторым общим свойством позволяющим объеденить их в одно целое.    
            \end{definition}
            

            \textsc{Обозначения:}
            \begin{itemize}
                \item $A, B, C$ - множества, $a,b,c$ - элементы множества 
                \item $\forall$ - квантор общности (для каждого)
                \item $\exists$ - найдется
                \item $\mathbb{X/E/U}$ - универсальные множества 
                \item $\emptyset$ - пустое множество 
                \item ! - единственность 
                \item $\rightarrow$ - следовательно 
            \end{itemize}

            \textsc{Операции над множествами:}
            \begin{itemize}
                \item $A  \cap B$ - объединение множеств (коммутативно и ассоциативно)
                \item $A \cup B$ - пересечение множеств (коммутативно и ассоциативно)
                \item $ A \backslash B $ - разность множеств 
                \item $\bar{A}$ - отрицание
                \item $A \varDelta B$ - симметрическая разность $(A \cup B) \backslash (A \cap B)$
                \item $A \times B = \{(x,y) | x \in A, y \in B \}$ - декартово произведение
            \end{itemize}

        \section{Отображения (функция)}
        \begin{definition}
            правило по которому $\forall x \in D \exists !y \in V$ 
        \end{definition}
            $F: D$ (область определения) $\rightarrow $ (правило перевода) $V $  (область значений)

            \textsc{Вариативности функциональных отображений:}
            \begin{itemize}
                \item Сюръекция ($\forall y \in V: \exists x \in D$) - каждый элемент в области значений функции имеет прообраз в области определения
                \item Инъекция ($ \forall y \in V : \exists ! x \in D$) - каждый элемент в области определения функции имеет образ в области значений. Не каждый образ имеет прообраз. 
                \item Биекция ($ \forall F : A \rightarrow B \exists ! F^{-1} : B \rightarrow A $ ) - функция яаляется и сюръекцией и биекцией. 
            \end{itemize}
        
        \section{Характеристики множеств}
            \begin{definition}
                Мощность (кардинальное число) -  количество различных элементов множества. 
            \end{definition}

            \begin{definition}
                Эквивалентность множеств: множества эквивалентны $(A \sim B)$ если они равномощны. $\forall x \in X \exists ! y \in Y$ и $\forall y \in Y \exists ! x \in X$
            \end{definition}
                
            \begin{definition}
                Счетность множеств: множество счетно (исчислимо), если $A \sim \mathbb{N}$ 
            \end{definition}

            \begin{definition}
                Мощность континуума: множество эквивалентное множеству точек отрезка $[0,1]$ имеет мощность континуума.
            \end{definition}
            
            \begin{theorem}
                Множество всех точек отрезка $[0;1] $ - несчетно
            \end{theorem}

            \begin{theorem}[Кантора-Бернштейна]
                Если $A \sim B' (B' \subset  B) $ и $B \sim A' (A' \subset A) \Rightarrow A \sim B$\\
                Если $A \subset B \subset C$, причем $A \sim C \Rightarrow A \sim B$
            \end{theorem}

            \begin{definition}[Сравнение мощностей множеств]
                $\exists B' \in B: B' \sim A$ и $\nexists A' \in A: A' \sim B \Rightarrow |A| < |B|$
            \end{definition}
        
        \section{Множества чисел}
        \begin{itemize}
            \item $\mathbb{N}$ - натуральные числа $\{1,2,3...\}$
            \item $\mathbb{Z} $ - целые числа $\{-1, 0, 1, 2..\}$ 
            \item $\mathbb{Q}$ - рациональные числа $\{\frac{2}{3},0.(3)\}$
            \item $\mathbb{R}$ - вещественные (действительные числа) $\{\sqrt{2}, \pi, e \}$
            \item $\mathbb{C} $ - комплексные 
        \end{itemize}
        $\mathbb{N} \subset \mathbb{Z} \subset \mathbb{Q} \subset \mathbb{R} \subset \mathbb{C}$
        
        \textsc{Основные свойства вещественных чисел:}
        \begin{itemize}
            \item Транзитивность ($a>b, b>c \rightarrow a > c$)
            \item Ассоциативность $(a+(b+c) = (a+b)+c)$
            \item Коммутативность $a+b = b+a$
            \item Дистрибутивность $(a+b)\cdot c = a \cdot c + b \cdot c$
            \item $\forall a,b \in \mathbb{R} \exists ! c \in \mathbb{R}: a+b=c$
            \item $\forall a \neq 0 \exists! a^{-1}: a \cdot a^{-1} = 1$
        \end{itemize}
            
        \textsc{Грани множеств:}

        \begin{itemize}
            \item $\forall b \in \mathbb{R}:\forall a \in A \rightarrow a \leq b$ - верхняя грань 
            \item $\forall d \in \mathbb{R}:\forall a \in A \rightarrow d \leq a$ - нижняя грань 
        \end{itemize}

            Грани не единственны

        \begin{definition}
            Точная верхняя/нижняя грань - минимальный/максимальный элемент множества верхних/нижних граней множеств. 
        \end{definition}


        \textsc{Свойство точной верхней грани:}

        Если $ b = \sup A$, то $\forall \epsilon > 0 \exists a \in A: a > b - \epsilon$

        $\blacktriangleright$
            Допустим обратное. Тогда $a \leq b - \epsilon$ А это невозможно т.к b является наименьшей верхней гранью.
        $\blacktriangleleft$
    
        \textsc{Свойство нижней верхней грани:}

        Если $ d = \inf A$, то $\forall \epsilon > 0 \exists a \in A: a < d + \epsilon$

        $\blacktriangleright$
            Док-во аналогично свойству точной верхней грани. 
        $\blacktriangleleft$

        \begin{theorem}[Принцип вложенных отрезков]
            Пусть $\{[a_n, b_n]\}^{\infty}_{n=1}: \forall n \in \mathbb{N} \rightarrow [a_{n+1}, b_{n+1} \subset [a_n, b_n]]$ тогда $\exists ! c \in \mathbb{R}: \forall n \in \mathbb{N} \rightarrow c \in [a_n, b_n]$
        \end{theorem}
        $\blacktriangleright$
            Пусть длина отрезка - $d(n) = b_n - a_n. \forall k \in \mathbb{N} \rightarrow d(1) > d(k)$. Пусть $c := \sup a_n \Rightarrow \forall n \rightarrow a_n \leq c \leq b_n. \forall n \rightarrow c \leq b_n \Rightarrow c \in [a_n, b_n]$. Единственность точки следует из стремления длин отрезков к нулю.
        $\blacktriangleleft$

    \section{Метод математической индукции}
        Для обоснования ММИ используем свойство натуральных чисел: $\forall A \subset \mathbb{N}: A \neq \emptyset \exists a' \in A: \forall a \in A \rightarrow a' \leq a$. Метод математической индукции для док-ва утверждения на множестве $A$ состоит из шагов:
        \begin{itemize}
            \item База индукции - проверяем справедливость на $a'$
            \item Индукционное предположение - проверяем для произвольного элемента $a_k \in A$
            \item Индукционный шаг - доказываем справедливость для $a_{k+1} \in A$
        \end{itemize}

    \section{Бином Ньютона}
        \begin{equation}
            (1+x)^n = \sum^n_{k=0} C^k_n x^k
        \end{equation}
        $C^k_n =\binom{n}{k}= \frac{n!}{k!(n-k)!}$ - биноминальный коэффициент\\
        $\binom{n}{k} + \binom{n}{k+1} = \binom{n+1}{k+1}$

        $\blacktriangleright$
            По методу математической индукции. 
            При $n=1. 1+x = C^0_1 + C^1_1 x = 1+x$
            При $n=t$ формула так же верна. \\ 
            При $n=t+1 \\ (1+x)^{t+1} = (1+x)^t(1+x) = \binom{t}{0}x^0 + ... + \binom{t}{t}x^t + \binom{t}{0} x + ... + \binom{t}{t}x^{t+1} = \binom{t+1}{0} + \binom{t+1}{1}x + ... + \binom{t+1}{t+1} x^{t+1}$
        $\blacktriangleleft$

    \section{Неравенство Бернулли}
        \begin{equation}
            (1+x)^n > 1 + xn
        \end{equation}
        При $x>-1, x\neq 0, n \geq 2$ \\ 
        Док-во по ММИ.


\chapter{Пределы}
        \section{Числовая последовательность и ее свойства}
            \begin{definition}[Числовая последовательность]
                
                $\\\beth x_n = f(n), f:\mathbb{N} \rightarrow \mathbb{R}$
            \end{definition}

            Операции с числовыми последовательностями выполняются почленно. 
        
                \begin{definition}[Ограниченность последовательности]
                    $\\ \exists A \in \mathbb{R} : \forall n \in \mathbb{N} \rightarrow |x_n| \leq A$
                \end{definition}


                \begin{definition}[Бесконечно большая последовательность]
                    Последовательность называется бесконечно большой, если множество членов удовлетворяющих условию $|x_n| \leq c$ конечно.
                    $\\ \forall c > 0 \exists n(c) \in \mathbb{N} : \forall n > n(c) \rightarrow |x_n| > c$
                \end{definition}
                
                \begin{definition}[Бесконечно малая последовательность]
                    Последовательность называется бесконечно малой, если множество членов удовлетворяющих условию $|x_n| \geq c$ конечно.
                    $\\ \forall c > 0 \exists n(c) \in \mathbb{N} : \forall n > n(c) \rightarrow |x_n| < c$
                \end{definition}

                \begin{theorem}[Ограниченность бесконечно малой последовательности]
                    Если ${x_n}$ - б.м.п $\Rightarrow \forall n \in \mathbb{N} \rightarrow |x_n| < C, C \in \mathbb{R}_+$
                \end{theorem}
                $\blacktriangleright$
                    По определению бесконечно малой последовательности, кол-во элементов $|x_n| \geq C$ конечно. Возьмем $C = max(|x_1|, |x_2|, ... , |x_n|)$. Получим $\forall n \in \mathbb{N} \rightarrow |x_n| < C$
                $\blacktriangleleft$
                
                \begin{theorem}[Арифметика бесконечно малых последовательностей]
                    $\\$
                        \begin{itemize}
                            \item Если $\{x_n\}$ - б.м.п, то и $\{|x_n|\}$ - б.м.п
                            \item Сумма/разность конечного кол-ва б.м.п - б.м.п
                            \item Произведение б.м.п на ограниченную последовательность - б.м.п $\blacktriangleright$ 
                                $\{x_n\} - $ б.м.п; $\{y_n\} - $  ограниченная последовательность, $C$ - грань $\{y_n\}$
                                $\forall \epsilon > 0: \exists N : \forall n \in \mathbb{N} => |x_n| < \frac{\epsilon}{M} \Rightarrow |x_n \cdot y_n| = |x_n| \cdot |y_n| < \frac{\epsilon}{M} \cdot M = \epsilon$
                            $\blacktriangleleft$
                        \end{itemize}
                \end{theorem}
            \section{Предел числовой последовательности}
                \begin{definition}[Сходщаяся последовательность]
                    Последовательность $\{x_n\}$ называется сходящейся (имеющей предел), если $\forall \epsilon > 0 \exists n(\epsilon) \in \mathbb{N} : \forall n > n (\epsilon) \rightarrow |x_n - a | < \epsilon$ или же $x_n \in U_\epsilon (a)$
                \end{definition}

                \begin{theorem}[О единственности предела последовательности]
                    Если последовательность сходится, то она имеет единственный предел.
                \end{theorem}
                $\blacktriangleright$ Предположим обратное
                    $\\ \lim_{n\rightarrow\infty}x_n = a \\ \lim_{n\rightarrow\infty}x_n = b \\ $
                    где $a>b, \epsilon = \frac{a-b}{2} > 0$, тогда $\forall n > n(\epsilon) \rightarrow a - \epsilon < x_n < b+\epsilon$ Получается, что $a-\epsilon = b+\epsilon = \frac{a+b}{2}$
                $\blacktriangleleft$

            \section{Свойства сходящихся последовательностей}
            \begin{itemize}
                \item Предел б.м.п = 0
                \item Сходящаяся последовательность ограничена 
                \item Не всякая сходящаяся последовательность является ограниченной
                \item $\lim_{n\rightarrow\infty}(x_n \pm y_n) = \lim_{n\rightarrow\infty}x_n \pm \lim_{n\rightarrow\infty}y_n$
                \item $\lim_{n\rightarrow\infty}(x_n \cdot y_n) = \lim_{n\rightarrow\infty}x_n \cdot \lim_{n\rightarrow\infty}y_n$
                \item $\forall n \in \mathbb{N}: y_n \ne 0,  \lim_{n\rightarrow\infty}y_n \ne 0 \Rightarrow  \lim_{n\rightarrow\infty} \frac{x_n}{y_n} = \frac{ \lim_{n\rightarrow\infty}x_n }{ \lim_{n\rightarrow\infty}y_n } $
            \end{itemize}

            \begin{theorem}
                Если $\lim_{n\rightarrow\infty} x_n = a \neq 0 \Rightarrow \exists n' \in \mathbb{N}: \forall n > n' \rightarrow |x_n| > \frac{1}{2}|a|$ при $a>0$, и $x_n < \frac{1}{2}|a|$ при $a < 0$
            \end{theorem}

            $\blacktriangleright$
                $|x_n - a| < \frac{1}{2}|a| \Leftrightarrow a - \frac{1}{2}|a| < x_n < a + \frac{1}{2}|a|$
            $\blacktriangleleft$

            \begin{theorem}[О сравнении пределов]
                $\forall n \in \mathbb{N} \rightarrow x_n \leq y_n \Rightarrow \lim_{n\rightarrow\infty}x_n \leq \lim_{n\rightarrow\infty}y_n$
            \end{theorem}
            $\blacktriangleright$ 
                $a\leq b$ Предположим обратное, $a>b$. Пусть $\epsilon = \frac{a-b}{2}$ Получаем $x_n > a - \epsilon \rightarrow x_n > \frac{a+b}{2}, y_n < b + \epsilon \rightarrow y_n < \frac{a+b}{2}$
            $\blacktriangleleft$

            \begin{theorem}[о двух миллиционерах]
                 Пусть $\lim_{n\rightarrow\infty}x_n = \lim_{n\rightarrow\infty}y_n = a$, тогда $\forall n \in \mathbb{N} \rightarrow x_n \leq y_n \Rightarrow \forall \{z_n\}: \forall n \rightarrow x_n \leq z_n \leq y_n$ справедливо, что $\lim_{n\rightarrow\infty}z_n = a$
            \end{theorem}
            $\blacktriangleright$
                $a - \epsilon < x_n \leq z_n \leq y_n < a + \epsilon \rightarrow z_n \in (a-\epsilon, a+\epsilon)$
            $\blacktriangleleft$

            \section{Монотонная последовательность}

            \begin{definition}[Монотонная последовательность]
                Последовательность называется возрастающей $(\{x_n\} \uparrow)$, если $\forall n \in \mathbb{N} \rightarrow x_{n+1} \geq x_n$, убывающей, если $x_{n+1} \leq x_n$. В случае выполнения строгого неравенства, называется строго возрастающей/убывающей.
            \end{definition}

            \begin{theorem}[Вейерштрасса]
                    Если неубывающая/невозрастающая последовательность ограничена сверху/снизу, то она сходится к $sup x_n$
            \end{theorem}
            $\blacktriangleright$ 
                Пусть $C$ - верхняя граница последовательности. Тогда $\exists x_k > C-\epsilon$. Так как последовательность является монотонной, то все члены последовательности начиная с $x_k$ удовлетворяет неравенству $C-\epsilon < x_k \leq C \Rightarrow \lim_{n\rightarrow\infty} x_n = C$
            $\blacktriangleleft$  
            
            \begin{definition}[Эйлерова константа]
                \begin{equation}
                    e:= \lim_{n\rightarrow\infty} (1 + \frac{1}{n})^n
                \end{equation}
            \end{definition}

            \begin{theorem}[Теорема Штольца]
                $\{y_n\} $ - неограниченная неубывающая последовательность. Если существует предел $\lim_{n\rightarrow\infty} \frac{x_{n+1} - x_n}{y_{n+1} - y_n} = l$, то тогда существует $\lim_{n\rightarrow\infty} \frac{x_n}{y_n} = l$
            \end{theorem}
            $\blacktriangleright$ 
                $\lim_{n\rightarrow \infty } \frac{x_{n+1}-x_n}{y_{n+1}-y_n} = l \Rightarrow \frac{x_{n+1}-x_n}{y_{n+1}-y_n} = l + a_n$, где $a_n$ - б.м. $\Rightarrow \forall \epsilon > 0 \exists n(\epsilon) : \forall n \geq n(\epsilon) \rightarrow |a_n| < \frac{\epsilon}{2}$. Домножим на $(y_{n+1} - y_n)$. \\ $x_{n+1} - ly_{n+1} = x_n - ly_n + a_n (y_{n+1} - y_n)$
                \\ Сложим все равенства от$n(\epsilon)+1$ до $n+1$\\ 
                $x_{n+1} - ly_{n+1} = x_{n(\epsilon)} - ly_{n(\epsilon)}+ \sum_{i=n(\epsilon)}^n a_i (y_{i+1} - y_i)$
                Получается, что: \\ 
                $|x_{n+1} - ly_{n+1}| \leq |x_{n(\epsilon)} - ly_{n(\epsilon)}|+|a_{n(\epsilon)} \cdot |y_{n(\epsilon)} - y_{n(\epsilon)}| \dots |a_n|\cdot |y_{n+1} - y_n|$ \\
                Так как $|a_n| < \frac{\epsilon}{2}$ , получается: \\ 
                $|x_{n+1} - ly_{n+1}| \leq |x_{n(\epsilon)} - ly_{n(\epsilon)}|+\frac{\epsilon}{2} |y_{n(\epsilon)} - y_{n(\epsilon)}| \dots\frac{\epsilon}{2} |y_{n+1} - y_n| \qquad | : (y_{n+1})$  \\ 
                $|\frac{x_{n+1}}{y_{n+1}} - l| < \frac{|x_{n(\epsilon) - ly_{n(\epsilon)}}|}{y_{n+1}} + \frac{\epsilon}{2} \frac{y_{n+1} - y_{n(\epsilon)}}{y_{n+1}}$. \\ 
                Так как ${y_n}$ стремится к $+\infty \Rightarrow \exists n_1(\epsilon):\forall n > n_1 \rightarrow \frac{|x_{n(\epsilon) - ly_{n(\epsilon)}}|}{y_{n+1}} < \frac{\epsilon}{2}$ \\ 
                Полагая $n_0 = max{n_1, n(\epsilon)}$ получаем, что $\forall n > n_0 \rightarrow |\frac{x_{n+1}}{y_{n+1} } - l| < \epsilon$, следовательно $\frac{x_n}{y_n} \rightarrow l$ при $n\rightarrow\infty$

            $\blacktriangleleft$


            \begin{definition}
                Подпоследовательность - последовательность вида $y_n = x_{k_n}$, гдe $\{x_n\}$ - некая последователньость, а $\{k_n\} $ - некая строго возрастающая последовательность натуральных чисел.
            \end{definition}

            Свойства подпоследовательностей:
            \begin{itemize}
                \item Если последовательность сходится к $a$, то все ее подпоследовательности тоже сходятся к $a$
                \item Все подпоследовательности, как и сама последовательность, если они сходятся, то сходятся к одному и тому же пределу.
            \end{itemize}


            \begin{definition}
                Точка $a \in \bar{\mathbb{R}} = \mathbb{R} \cup \{\pm \infty\}$ называется предельной точкой последовательности, если $\forall \epsilon > 0 в U(a, \epsilon)$ содержится бесконечно много элементов этой последовательности. 
            \end{definition}

            \begin{definition}
                Точка $a \in \bar{\mathbb{R}} = \mathbb{R} \cup \{\pm \infty\}$ называется предельной точкой последовательности, если из этой последовательности можно выделить подпоследовательность сходящуюся к $a$.
            \end{definition}

            \begin{definition}
                Наибольшая предельная точка последовательности называется верхним пределом, а наименьшая - нижний
            \end{definition}

            \begin{theorem}
                У всякой ограниченной последовательности существуют верхний и нижний пределы и хотя бы одна предельная точка.
            \end{theorem}

            \begin{theorem}[теорема Больцано-Вейерштрасса]
                Из всякой ограниченной последовательности можно выделить сходящуюся подпоследовательность.
            \end{theorem}

            \begin{theorem}[критерий Коши]
                Для сходимости последовательности необходимо и достаточно, что бы она была фундаментальной (выполнялось условие Коши) \\
                \textsc{Условие Коши:} $\forall \epsilon > 0 \exists n(\epsilon) \in \mathbb{N} : \forall n,p \in \mathbb{N}: n, m > n(\epsilon) \rightarrow |x_n - x_m| < \epsilon$
            \end{theorem}

            $\blacktriangleright$
                \textsc{Необходимость:} $\forall \epsilon > 0 \exists n(\epsilon) \in \mathbb{N}: \forall n > n(\epsilon) \rightarrow |x_n - a| < \frac{\epsilon}{2}$, теперь если $m,m>n(\epsilon)$, то $|x_n-x_m| = |x_n -a = (x_m - a) \leq |x_n -a| + | x_m - a| < \frac{\epsilon}{2} + \frac{\epsilon}{2} < \epsilon$ \\
                \textsc{Достаточность: } $|x_n - x_{n(\epsilon)}| < \epsilon \Rightarrow |x_n| \leq |x_{n(\epsilon)} + \epsilon|$. Из этого следует что последовательность ограничена. По Т. Больцано-Вейерштрасса из нее можно выделить сходящуюся подпоследовательность. $|x_n - x_{k_n}| < \frac{\epsilon}{2}$, где $x_{k_n}$ - элемент сходящейся подпоследовательности. Переходя к пределу при $k\rightarrow\infty$ получаем $|x_n - a| < \frac{\epsilon}{2}$ 
            $\blacktriangleleft$
        
        \chapter{Предел функции}

            \begin{definition}[предел функции в точке по Гейне]
            $\\ \forall \{x_n\}_{n=1}^{\infty}:(x_{n} \xrightarrow{n\rightarrow\infty}a $ и $\forall n \in \mathbb{N} \rightarrow x_n \neq a) \rightarrow f(x_n) \xrightarrow{n\rightarrow\infty}b$
            \end{definition}

            \begin{definition}[предел функции в точке по Коши]
                $\\ \forall \epsilon > 0 \exists \delta(\epsilon) > 0: \forall x : 0 < |x-a| < \delta \rightarrow |f(x) - b| < \epsilon$
            \end{definition}

            \begin{theorem}
                Определения по Коши и Гейне являются эквивалентными. 
            \end{theorem}
            $\blacktriangleright$ понять и написать док-во$\blacktriangleleft$


            \textsc{Свойства пределов:}
            \begin{itemize}
                \item $\lim_{x\rightarrow a} (f(x) \pm g(x)) = \lim_{x\rightarrow a} f(x) \pm \lim_{x\rightarrow a} g(x)$
                \item $\lim_{x\rightarrow a} (f(x) \cdot g(x)) = \lim_{x\rightarrow a} f(x) \cdot \lim_{x\rightarrow a} g(x)$
                \item $\lim_{x\rightarrow a} \frac{f(x) }{g(x)} = \frac{\lim_{x\rightarrow a} f(x)}{\lim_{x\rightarrow a} g(x)}$
            \end{itemize}

            \begin{definition}[односторонний предел по Гейне]
                $\\ \forall \{x_n\}_{n=1}^{\infty}:(x_{n} \xrightarrow{n\rightarrow\infty}a $ и $\forall n \in \mathbb{N} \rightarrow x_n > a) \rightarrow f(x_n) \xrightarrow{n\rightarrow\infty}b$
                $\\ x_n > a $ для правостороннего и $x_n < a $ для левостороннего
            \end{definition}

            \begin{definition}[односторонний предел по Коши]
                $\\ \forall \epsilon > 0 \exists \delta(\epsilon) > 0: \forall x \in (a, a+\delta): \delta \rightarrow |f(x) - b| < \epsilon$
                $\\ \forall x \in (a, a+\delta)$ для правостороннего предела и $ \forall x \in (a-\delta, a)$ для левостороннего предела
            \end{definition}

            \begin{theorem}
                Если у функции $f(x)$ правосторонний и левосторонний предел в точке $a$, то она имеет предел в точке $a$
            \end{theorem}

            \begin{theorem}[критерий Коши]
                Для существования конечного предела функции в точке $a$ необходимо и достаточно что бы выполнялось условие Коши: $\forall \epsilon > 0 \exists \delta = \delta(\epsilon) > 0: \forall x', x'' \in U(a, \delta) \rightarrow |f(x') - f(x'')| < \epsilon$
            \end{theorem}

            $\blacktriangleright$ 
                \textsc{Необходимость:} Пусть $b$ - предел функции $f(x)$, тогда: $|f(x')-b| < \frac{\epsilon}{2}$ и $f(x'')-b| < \frac{\epsilon}{2} \Rightarrow |f(x') - f(x'')| = |f(x') -b -f(x'') + b| \leq |f(x') - b| + |f(x'') -b| < \frac{\epsilon}{2} + \frac{\epsilon}{2}$   \\ 
                \textsc{Достаточность:} Согласно определению по Гейне, возьмем $x_n \in U(a): x_n \xrightarrow{n\rightarrow \infty} a$. Из условия Коши для последовательностей: $\forall \epsilon >0 \exists n(\epsilon) \in \mathbb{N}: \forall n \geq n(\epsilon) \rightarrow x_n \in U(a, \delta) \Rightarrow |f(x_n) - f(x_m)| < \epsilon, \forall n,m \geq n(\epsilon)$. Тогда последовательность $\{f(x_n)\}$ сходится в силу критерия Коши для последовательностей. 
            $\blacktriangleleft$

            
            \textsc{Модификация условия Коши}:
            \begin{itemize}
                \item Для левостороннего предела: $a-\delta < x', x'' < a$
                \item Для правостороннего предела: $a < x', x'' < a+\delta$
                \item Для $x\rightarrow \infty: |x'|, |x''| > \delta$ 
                \item Для $x\rightarrow -\infty: x', x'' <  -\delta$ 
                \item Для $x\rightarrow +\infty: x', x'' > \delta$ 
            \end{itemize}

        
            \begin{definition}
                Функция называется бесконечно малой, если ее предел равен нулю
            \end{definition}
            \begin{definition}
                Функция называется бесконечно большой в точке $a$ слева/справа, если ее односторонний предел равен $\pm \infty$
            \end{definition}

            \begin{definition}
                Функция $f(x)$ является в точке $a$ бесконечно малой более высокого порядка, чем $g(x)$, если $\lim_{x\rightarrow a}\frac{f(x)}{g(x)} = 0$. В таком случае это обозначается $f(x) = o(g(x)) $ при $x \rightarrow a$. \\ 
                $\exists U(a) : f(x) = g(x)c(x)$ при $x\rightarrow a$, где $c(x) \rightarrow 0$
            \end{definition} 

            \begin{definition}
                Симовлом О обозначают любую функцию $f(x) = O(g(x))$ ограниченную относительно $g(x)$ \\ 
                $|f(x)| \leq c |g(x)|$
            \end{definition}

            \textsc{Свойства для о/О:}
                \begin{itemize}
                    \item $o(c\cdot f(x)) = o(f(x)), c\neq 0$
                    \item $o(f(x)) \pm o(g(x)) = o(f(x))$
                    \item $o(f(x)) \cdot o(g(x)) = o(f(x)g(x))$
                \end{itemize}

            \textsc{Свойства для О:}
                \begin{itemize}
                    \item $O(o(f(x))) = o(f(x))$
                    \item $O(O(f(x))) = O(f(x))$
                    \item $o(O(f(x)))=o(f(x))$
                \end{itemize}

        \section{Эквивалентность}
                \begin{definition}
                    $f(x) \sim g(x)$, если $\lim_{x\rightarrow a} \frac{f(x)}{g(x)} = 1$
                \end{definition}

                \textsc{Эквивалентность при $x\rightarrow 0$: (при равенстве $+o(x)$)}
                \begin{itemize}
                    \item $\sin x \sim x$
                    \item $1 - \cos x \sim \frac{x^2}{2}$
                    \item $\tg x \sim x$
                    \item $\arcsin x \sim x$
                    \item $\arctg x \sim x$
                    \item $a^{b(x)} - 1 \sim b(x) \ln (a)$
                    \item $\ln(1+x) \sim x$
                    \item $(1+x)^{a} - 1 \sim ax$
                \end{itemize}

                \begin{definition}[Первый замечательный предел]
                    $$\lim_{x\rightarrow 0} \frac{\sin x}{x} = 1$$
                \end{definition}
                $\blacktriangleright$ разобрать и написать док-во$\blacktriangleleft$
            
        \section{Непрерывность функции}

        \begin{definition}[непрерывность по Гейне]
            $$\forall \{x_n\}: x_n \xrightarrow{n\rightarrow \infty} a, \{f(x_n)\} \xrightarrow{n\rightarrow\infty} f(a)$$
        \end{definition}

        \begin{definition}[непрерывность по Коши]
            $$\forall \epsilon > 0 \exists \delta(\epsilon) > 0 : \forall x \in D(f): |x-a| < \delta \rightarrow |f(x) - f(a)| < \epsilon$$
        \end{definition}

        \begin{definition}[формальное определение односторонней непрерывности]
            Функция непрерывна справа/слева если правый/левый предел равен $f(a)$
        \end{definition}

        \begin{theorem}[непрерывноть функций над арифметическими операциями]
            $f(x), g(x)$ - непрерывны в точке $a$. Тогда $f(x) \pm g(x), f(x) \cdot g(x), \frac{f(x)}{g(x)}$ непрерывны в точке $a$.          
        \end{theorem}

        $\blacktriangleright$ Так как функции в точке $a$ имеют пределы, соответственно равныы $f(a), g(a)$ то существуют пределы $f(a) \pm g(a) $ и тд. Т.к предел равен значению, то значит по определению эти функции непрерывны в точке $a$ $\blacktriangleleft$

        \begin{theorem}[о непрерывности сложной функции]
            $x = \phi(t)$ непрерывна в точке $a$, а функция $y=f(x)$ непрерывна в точке $b=\phi(a) \Rightarrow y=f(\phi(t))$ непрерывна в точке $a$
        \end{theorem}

        $\blacktriangleright$
            Пусть $\{t_n\}$ - последовательность значений сложной функции сходящейся к $a$.
            Так как $x=\phi(t)$ непрерывна и сходится к $a \Rightarrow $ последовательность аргументов сходится к $b=\phi(t)$ по определению по Гейне. 
            Так как $y=f(x)$ непрерывна в точке $b=\phi(\epsilon)$ и $\{x_n\}$ сходится к $b=\phi(a)$ и является последовательностью значений аргументов, то соотсветствующая последовательность функции $f=[\phi(t)]$ сходится к $f(b) = f[\phi(a)]$
        $\blacktriangleleft$


        \begin{theorem}[существование односторонних пределов монотонной на отрезке функции]
            Если функция $f(x)$ монотонна на отрезке $[a,b]$, то у нее существует правый и левый предел в любой внутренней точке отрезка, так же существует правый предел в точке $a$ и левый предел в точке $b$.
        \end{theorem}
        $\blacktriangleright$
            Т.к функция монотонна, то $\forall x \in [a; x_n] \rightarrow f(x) \leq f(x_0)$. Т.к множество значений функции ограниченой сверху, то по теоереме о точной верхней грани существует $\sup ((f(x)) = M \leq f(x_9)) \Rightarrow \forall \epsilon > 0 \exists \delta = x_0 - x_\epsilon > 0: \forall x \in (x_0 - \delta; x_0) \rightarrow |M - f(x_0)| < \epsilon$
        $\blacktriangleleft$


        \begin{theorem}[непрерывность монотонной функции]
            Для того что бы монотонная функция являлась непрерывной на отрезеке $[a,b]$, необходимо и достаточно что бы лбое число $c: f(a)<c<f(b)$ было значением этой функции
        \end{theorem}

        \begin{theorem}[монотонность и непрерывность обратной функции]
            Если функция монотонна на отрезке $[a,b]$ и непрерывна на нем, то на этом отрезке определена обратная функция которая так же непрерывно возрастает/убывает на данном отрезке.
        \end{theorem}
        $\blacktriangleright$
            Так как функция монотонна и непрерына на отрезке $\Rightarrow$ по теореме непрерывности монотонной функции множеством значений функции является отрезок $[f(a), f(b)] \Rightarrow $ существует обратная функция в силу биективности правила $f$. 
        $\blacktriangleleft$

        \begin{definition}[устранимый разрыв]
                $\exists \lim_{x\rightarrow a}f(x)$, но $a \notin D(f)$ или $f(a) \ne \lim_{x\rightarrow a } f(x)$. В случае устранимого разрыва функцию можно доопределить не меняя значений функции в других точках
        \end{definition}

        \begin{definition}[разрыв первоо рода]
            $$\lim_{x\rightarrow a-}f(x) \ne \lim_{x\rightarrow a+} f(x)$$
        \end{definition}

        \begin{definition}[разрыв второго рода]
            Хотя бы один из односторонних пределов не существует или бесконечен.
        \end{definition}

        \begin{definition}[кусочно-непрерывная функция на отрезке]
            Функция определена всюду на отрезке и непрерывна во всех внутренних точках, кроме ограниченного числа точек в которых имет разрывы первого рода.
        \end{definition}

        \begin{definition}[Ограниченность функции]
            $\exists M, m \in \mathbb{R} : \forall x \in X \rightarrow m \leq f(x) \leq M$
        \end{definition}

    \chapter{Дифференциальное исчисление}
    Пусть $y = f(x)$ задана на $(a,b)$, рассмотрим $x_0 \in (a,b)$ и приращение аргумента $\triangle x$ - произвольное число $x_0 + \triangle x \in (a,b)$
    \begin{definition}[приращение функции]
        $$\triangle y = \triangle f = f(x_0 + \triangle x) - f(x_0)$$
    \end{definition}

    \begin{definition}[производная]
        Функция имеет производну в точке, если существует предел

        $$ \lim_{\triangle x \rightarrow 0} \frac{\triangle y}{\triangle x} = \lim_{\triangle x \rightarrow 0} \frac{f(x_0 + \triangle x) - f(x_0)}{\triangle x} = f'(x_0)$$
    \end{definition}

    \begin{theorem}[о непрерывности функции имеющей производную]
        Если функция имеет производную в некоторой точке, то функция непрерывна в этой точке.
    \end{theorem}
    $\blacktriangleright$По определению производной существует предел. Из его существования следует, что в достаточно малой окрестности справедливо равененство $\frac{\triangle y}{\triangle x} = f'(x_0) + a(\triangle x)$, где $a(\triangle x)$ - б.м.ф. при $\triangle x \rightarrow 0$. Получаем $\triangle y = f'(x_0)\triangle x + a(\triangle x ) \triangle x \Rightarrow \triangle y \xrightarrow{\triangle x \rightarrow 0} 0$, следовательно $y=f(x)$ непрерывна в точке. $\blacktriangleleft$

    \begin{definition}[односторонние производные]
        Если существует односторонний предел: $\lim_{\triangle x \rightarrow 0 +/-} \frac{\triangle y}{\triangle x}$, то этот предел называют односторонней производной и обозначают $f_{+/-}'(x)$ 
    \end{definition}  

    \begin{theorem}[Основные правила дифференцирования] 
        $\\$
        \begin{itemize}
            \item $(u \pm v)' = u' \pm v' $
            \item $(u \cdot v)' = u' \cdot v + u \cdot v'$
            \item $(\frac{u}{v})' = \frac{u' \cdot v - u \cdot v}{v^2}$
        \end{itemize}
    \end{theorem}
    $\blacktriangleright$ понять и написать док-во $\blacktriangleleft$


    \begin{definition}
        Если функция определена в окрестности точки $x$, то ее приращение можно представить в виде $\triangle y = \triangle f = f(x+ \triangle x) - f(x) = A\triangle x + o(\triangle x), \triangle x \rightarrow 0$
    \end{definition}

    \begin{theorem}
        Функция дифференцируема в точке только тогда, когда функция имеет в этой точке производную. Если функция дифференцируема, то $\triangle y = f'(x_0) + o(\triangle x), \triangle x \rightarrow 0$
    \end{theorem}

    \begin{definition}[о непрерывном дифференцировании]
        Функцию, производая которой непрерывна в точке или на промежутке, называют непрерывно дифференцируемой 
    \end{definition}

    \begin{definition}[Дифференциал]
        Если функция дифференцируема в точке, то линейную часть ее приращения называют дифференциалом
        $$df = f'(x)\triangle x, \triangle x = dx \Rightarrow dy=f'(x)dx \Rightarrow f'(x)=\frac{dy}{dx}$$
    \end{definition}

    \begin{definition}[Геометрический смысл производной]
        Производная - тангенс касательной к графику функции в точке
    \end{definition}

    \begin{theorem}
        Если функция в некоторой окрестности точки непрерывна и строго монотонна и имеет производную в точке отличную от нуля, то обратная функция имеет в этой точке производную и справедливо: $g'(x) = \frac{1}{f'(x)}$
    \end{theorem}

    $\blacktriangleright$
        Если функция строко возрастает, то ее приращения $\triangle y, \triangle x$ имеют одинаковые знаки. Переходя к пределу, видим что: $\lim_{\triangle y \rightarrow 0} \frac{\triangle x}{\triangle y} = \lim_{\triangle x \rightarrow 0} \frac{1}{\frac{\triangle y}{\triangle x} } =+ \infty$. Для убывающей функции предел равен $-\infty$. 
    $\blacktriangleleft$

    \begin{theorem}
        Пусть функция $y=f(x)$ имеет производную в точке $x_0, f(x_0) = y_0$ и функция $z=\phi(y)$ имеет проивзодную в точке $y_0$. Тогда сложная функция $z = \phi(f(x))$ имеет производную в точке $x_0$ и справедливо равенство: $z(x_0)' = \phi'(y_0)f'(x_0)$
    \end{theorem}
    $\blacktriangleright$понять и доказать $\blacktriangleleft$

    \begin{theorem}[формула Лейбница]
        Если у функций $u(x), v(x)$ существует в точке $x_0$ производные порядка $n = 1, 2...$, то в этой точке существует производная порядка $n$ произведения: $$(u(x_0) \cdot v(x_0))^{(n)} = \sum^n_{k=0} C^k_n u^{(k)}(x_0)\cdot v^{(n-k)}(x_0)$$
    \end{theorem}
    $\blacktriangleright$Док-во по ММИ. При $n=1 \rightarrow (u\cdot v)'=u' \cdot v u \cdot v'$ \\ Пусть формула справедлива для $n=m$, тогда для $n=m+1$ получаем $(uv)^{(m+1)}=\sum^m_{k=0}C^k_{k=0}(u^{(k)}v^{(m-k)})' = \sum^{m}_{k=0}C^{k}_{m}(u^{(k+1)} v^{(m-k)} + u^{(k)}v^{(m-k+1)}) = \sum^{m}_{i=1} (C^{i-1}_{m}+C^{i}_{m})u^{(i)} v^{(m-i+1)} + C^{m}_{m} u^{(m+1)}v^{(0)} + C^{0}_{m} u^{(0)} v^{(m+1)} = \sum^{m+1}_{i=0}C^{i}_{i=0} u^{(i)} \cdot v^{m+1-i}$ $\blacktriangleleft$

    \begin{theorem}
        Если $f'(x_0)>0$, то функция строго возрастает, а если $f'(x_0) < 0$, то функция убывает.
    \end{theorem}
    $\blacktriangleright$
    Пусть $f'(x) > 0$. Так как $f'(x_0) = \lim_{\triangle x \rightarrow 0} \frac{\triangle y}{\triangle x}$, то при достаточно малых $x$ имеем $\frac{\triangle y}{\triangle x} > 0 \Rightarrow \triangle y, \triangle x $ одного знака. 
    $\blacktriangleleft$

    \begin{theorem}[теорема Ферма]
        Если функция имеет производную в точке, то эта точка может быть точкой локального экстремума функции только если $f'(x_0) = 0$
    \end{theorem}
    $\blacktriangleright$Очевидно$\blacktriangleleft$

    \begin{theorem}[теорема Дарбу о промежуточных значениях]
        $f(x)$ дифференцируема на $[x_1; x_2] \Rightarrow \forall D \in [f'(x_1); f'(x_2)] \exists d \in [x_1; x_2] : D = f'(d)$
    \end{theorem}

    \begin{theorem}[теорема Ролля]
        Если $f(a) = f(b),$ то существует точка $\xi \in (a,b) : f'(\xi) = 0$
    \end{theorem}
    $\blacktriangleright$Очевидно$\blacktriangleleft$

    \begin{theorem}[теорема Лагранжа о среднем]
        Для функции существует $\xi \in (a,b): f(b) - f(a) = f'(\xi)(b-a) $
    \end{theorem}

    $\blacktriangleright$
        Подберем число $\lambda$, такое что бы $\phi(x) = f(x) - \lambda, \phi(a) = \phi(b) \Rightarrow f(a) - \lambda a = f(b) - \lambda b \Rightarrow \lambda = \frac{f(b) - f(a)}{b-a}$. По Т. Ролля существует точка $\xi, \phi'(\xi) = 0 $
    $\blacktriangleleft$

    \begin{theorem}[теорема Коши о среднем]
        Пусть функции $g(a) \ne g(b)$, их производные не равны нулю одновременно, тогда $\exists \xi \in (a,b): \frac{f(b) - f(a)}{g(b) - g(x)} = \frac{f'(\xi)}{g'(\xi)}$
    \end{theorem}

    $\blacktriangleright$
        $\phi(x) = f(x) - \lambda g(x) \Rightarrow f(a) - \lambda g(a) = f(b) - \lambda g(b) \Rightarrow \lambda = \frac{f(b) - f(a)}{g(b) - b(a)}, \phi(a) = \phi(b) \Rightarrow$ по Т. Ролля существует $\phi'(\xi) = 0 \Rightarrow f'(\xi) - \frac{f(b) - f(a)}{g(b) - g(a)}g'(\xi) = 0 \Rightarrow \phi'(\xi ) = f'(x) - \lambda g'(x) = 0 \Rightarrow f'(\xi) - \frac{f(b) - f(a)}{g(b) - g(a)} g'(\xi) = 0$ 
    $\blacktriangleleft$

    \begin{theorem}[следствие из теоремы лагранжа]
        Если производная функции во всех точках интервала равна нулю, то она постоянна на отрезке
    \end{theorem}

    \begin{theorem}[правило Лопиталя]
        $$\lim_{x\rightarrow a}\frac{f(x)}{g(x)} = \lim_{x\rightarrow a}\frac{f(x)'}{g(x)'} $$
    \end{theorem}

    \section{Формула Тейлора}

    \begin{theorem}[формула Тейлора с остаточным членом в форме Пеано]
        $$f(x) = \sum^n_{k=0}\frac{f^{(k)} (x_0)}{k!}(x-x_0)^{k} + o((x-x_0)^n)$$
    \end{theorem}

    \begin{theorem}[формула Тейлора с остаточным членом в форме Лагранжа]
        $$f(x) = \sum^n_{k=0}\frac{f^{(k)} (x_0)}{k!}(x-x_0)^{k} + \frac{f^{(n+1)}(\xi)}{(n+1)!}(x - x_0)^{n+1}$$
    \end{theorem}

    \begin{definition}[формула Маклорена]
        При $x_0 = 0$ формула Тейлора принимает следующую запись:
        $$f(x) = \sum^n_{k=0}\frac{f^{(k)} (0)}{k!}(x)^{k} + o(x^n)$$
    \end{definition}
\end{document}




% $\blacktriangleright$ $\blacktriangleleft$