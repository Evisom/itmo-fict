\documentclass[14pt]{extreport}
\usepackage{gost}
\usepackage{hyperref}
\usepackage{makecell}
\usepackage{ragged2e}
\justifying

\makeatletter
\@addtoreset{figure}{part}% Reset figure numbering at every part
\makeatother
\renewcommand{\thefigure}{\arabic{figure}}% Figure number is part.figure
\renewcommand{\thetable}{\arabic{table}}



%Тут можно вставить дополнительные пакеты

\begin{document}
\pagestyle{empty} %  выключаем нумерацию
\includepdf[pages=-,pagecommand={}]{title_page.pdf}

\pagestyle{plain} % включаем нумерацию
\tableofcontents
 



\intro\label{intro}



Целью данное практической работы является построение таблицы, отражающей мои карьерные ожидания. Данная практическая работа направлена на исследование и анализ рынка труда, анализ образовательной программы, составление, создание страницы с математическим текстом и формулами, а также оформление всего документа согласно ГОСТ 7.32.


\chapter{МАТЕМАТИЧЕСКИЙ ТЕКСТ\label{chapter1}}
\section{Пример оформления математического текста}



\begin{landscape}
\chapter{ОБЗОР РЫНКА ВАКАНСИЙ\label{chapter2}}

\section{Front-end разработчик}

Таблица 1 - Frontend разработчик
\begin{longtable}[H]{lp{1\linewidth}}
\caption{Frontend разработчик \label{table1}}


\centering

\begin{small}


    \begin{tabular}{|c|p{5,5cm}|p{6cm}|p{5cm}|p{5cm}|}
	\hline 
	\makecell{№ \\ п.п.} &	\makecell{Название должности,\\ ссылка} &	\makecell{Требования} & 	\makecell{Дисциплины \\ из учебного \\плана} &	\makecell{Преимущества и \\недостатки}  \\ 
	\hline 
	1	& Senior Frontend developer

(\url{https://goo.su/fUzWQ})
& •	Знание JavaScript, HTML, CSS

•	Знание Angular, TypeScript

 &	•	Web-программирование
 
•	Программирование
 & + Гибкий график

+ Высокая заработная плата

 \\
	\hline
	2 & Senior Frontend developer (React)
	
(\url{https://goo.su/i9QwFZz})
& •	Знание JavaScript, HTML, CSS

•	Знание React, TypeScript, Redux
&	•	Web-программирование
 
•	Алгоритмы и структуры данных
&+	Официальное трудоустройство 

+ Гибкий график

- Заработная плата не такая большая\\


	\hline
	3 & Middle Frontend разработчик
	
(\url{https://goo.su/taUjVfT}) 
& •	Знание JavaScript, HTML, CSS

•	Знание VueJS, GraphQL
&
•	Информатика

•	Программирование
&
+	Не требуют много опыта

-	Полная занятость
\\
	\hline


    \end{tabular}
\end{small}
\end{longtable}

\addtocounter{table}{-1}
\newpage
Продолжение таблицы 1
\begin{longtable}[H]{lp{1\linewidth}}
\caption{Продолжение таблицы 1}



\centering
\begin{small}


    \begin{tabular}{|c|p{5,5cm}|p{6cm}|p{5cm}|p{5cm}|}
	\hline 
	\makecell{№ \\ п.п.} &	\makecell{Название должности,\\ ссылка} &	\makecell{Требования} & 	\makecell{Дисциплины \\ из учебного \\плана} &	\makecell{Преимущества и \\недостатки}  \\ 
	\hline 
4 & Frontend разработчик

(\url{https://goo.su/N1SU3Vi}) 
 & •	Знание JavaScript, HTML, CSS 

•	Знание React, Redux, GrapQL &
•	Алгоритмы и структуры данных &
+	Комфортный офис

+	Гибкий график

-	Полная занятость\\
\hline
5 & Frontend разработчик 

(\url{https://goo.su/KdSg}) & •	Знание JavaScript, React, Redux & •	Web-программирование & 
+	Офис в центре Петербурга

-	Нужен опыт работы\\

	\hline

    \end{tabular}
    \end{small}
\end{longtable}

Вывод: данная профессия имеет высокий спрос на рынке труда, хорошо оплачивается. Меньший порог входа и большое количество вакансий с частичной или проектной занятостью позволяет совмещать работу и учебу. Освоив дисциплины из моей образовательной программы, я, вероятно смогу получить эту профессию.




\newpage
\section{Back-end разработчик}
Таблица 2 – Backend разработчик
\begin{longtable}[H]{lp{1\linewidth}}

\caption{Backend разработчик}


\centering

\begin{small}


    \begin{tabular}{|c|p{5,5cm}|p{6cm}|p{5cm}|p{5cm}|}
	\hline 
	\makecell{№ \\ п.п.} &	\makecell{Название должности,\\ ссылка} &	\makecell{Требования} & 	\makecell{Дисциплины \\ из учебного \\плана} &	\makecell{Преимущества и \\недостатки}  \\ 
	\hline 
	1	& Senior Java / Kotlin developer
	
(\url{https://goo.su/hzrLf}) &
•	Знание Java, Kotlin

•	Понимание модели OSI, HTTP &
•	Прикладное программирование

•	Разработка приложений на Java &
+	Высокая заработная плата


+	Комфортный офис

 \\

	\hline
	2	& Java Developer
	
(\url{https://goo.su/UZgR}) &
•	Знание Java, Java SE, Spring

•	Опыт работы с SQL базами данных

&

•	Прикладное программирование

•	Разработка приложений на Java &
+	Гибкий график


-	Маленькая заработная плата\\
	\hline 
	3	& Java backend developer
	
(\url{https://goo.su/yjWIc}) &
•	Знание Java, Kotlin

•	Знание Java Spring &

•	Разработка приложений на Java &
+	Возможность удаленной работы

+	Гибкий график
 \\


	\hline


    \end{tabular}
    \end{small}
\end{longtable}




\addtocounter{table}{-1}
\newpage
Продолжение таблицы 2
\begin{longtable}[H]{lp{1\linewidth}}
\caption{Продолжение таблицы 2}

\centering

\begin{small}


    \begin{tabular}{|c|p{5,5cm}|p{6cm}|p{5cm}|p{5cm}|}
	\hline 
	\makecell{№ \\ п.п.} &	\makecell{Название должности,\\ ссылка} &	\makecell{Требования} & 	\makecell{Дисциплины \\ из учебного \\плана} &	\makecell{Преимущества и \\недостатки}  \\ 
	\hline 
	4	& Java/Kotlin developer
	
(\url{https://goo.su/QqoRDVJi}) &
•	Знание Java Spring SE, Kotlin 
•	Опыт работы с SQL базами данных 
& 
•	Разработка приложений на Java &
+	Высокая заработная плата

+	Удаленная работа

 \\

	\hline
5	& Senior Java Developer

(\url{https://goo.su/lsU1F}) &
•	Знание Java, Java SE, Spring

•	Понимание ООП, модели OSI
 &
•	Разработка приложений на Java

•	Алгоритмы и структуры данных &
+	Гибкий график

+	Возможность удаленной работы

-	Маленькая заработная плата \\


	\hline 


    \end{tabular}
    \end{small}
\end{longtable}
Вывод: данная профессия имеет меньший спрос на рынке труда, относительно предыдущей профессии, но как правило имеет большую заработную плату. На рынке труда меньше вакансий с проектной и частичной занятостью.  Освоив дисциплины из моей образовательной программы, я, вероятно смогу получить эту профессию.







\newpage
\section{DevOps инженер}
Таблица 3 – DevOps инженер
\begin{longtable}[H]{lp{1\linewidth}}
\caption{DevOps инженер \label{table3}}


\centering

\begin{small}


    \begin{tabular}{|c|p{5,5cm}|p{6cm}|p{5cm}|p{5cm}|}
	\hline 
	\makecell{№ \\ п.п.} &	\makecell{Название должности,\\ ссылка} &	\makecell{Требования} & 	\makecell{Дисциплины \\ из учебного \\плана} &	\makecell{Преимущества и \\недостатки}  \\ 
	\hline 
	1	& DevOps инженер
	
(\url{https://goo.su/wJxfQ}) &
•	Опыт работы с Linux, Docker

• Знание Python, SQL &
•	Информационная безопасность

•	Программирование &
+	Высокая заработная плата

+	Комфортный офис
 \\


	\hline
	2	& DevOps Engineer (Kubernetes)
	
(\url{https://goo.su/RnjXO}) &
•	Понимание архитектуры и принципов разработки web-приложений

•	Опыт работы с Linux, Kubernetes, SQL &
•	Информатика

•	Информационная безопасность &
+	Удаленная работа

-	Полная занятость
\\

	\hline 
	3	& DevOps Engineer (GCP/Jenkins)
	
(\url{https://goo.su/EM9s}) 	&
•	Знание Golang

•	Умение работать с Linux, Docker, k8 &
•	Информационная безопасность

•	Информатика &
+	Высокая заработная плата

+	

-	Маленькая кампания \\


	\hline


    \end{tabular}
    \end{small}
\end{longtable}

\newpage
Продолжение таблицы 3
\begin{longtable}[H]{lp{1\linewidth}}
\caption{Продолжение таблицы 3}

\centering

\begin{small}


    \begin{tabular}{|c|p{5,5cm}|p{6cm}|p{5cm}|p{5cm}|}
	\hline 
	\makecell{№ \\ п.п.} &	\makecell{Название должности,\\  ссылка} &	\makecell{Требования} & 	\makecell{Дисциплины \\ из учебного \\плана} &	\makecell{Преимущества и \\недостатки}  \\ 
	\hline 
	4	& DevOps инженер SRE 
	
(\url{https://goo.su/zB5JX}) &
•	Знание Golang, Python

•	Опыт работы с Linux, Docker, AWS &
•	Информационная безопасность

•	Информатика &
+	Высокая заработная плата

-	Офис далеко от метро
\\



	\hline
	5	& DevOps инженер 
	
(\url{https://goo.su/yuGzpG}) &
•	Знание Python

•	Умение работать с Linux, Docker
	&
•	Информатика

• Информационная безопасность & 
+	Гибкий график
 
+	Высокая заработная плата 
\\

	\hline 


    \end{tabular}
    \end{small}
\end{longtable}

Вывод: данная профессия набирает популярность на рынке труда в России. На рынке присутствует не так много вакансий, но все они очень хорошо оплачиваются. Нужно хорошо разбираться в нескольких областях сразу. Освоив дисциплины из моей образовательной программы, я, вероятно смогу получить эту профессию.
\end{landscape}



\conclusions

Практическая работа выполнена. Построена таблица желаемых должностей и оформлен математический текст согласно ГОСТ 7.32. Выполняя работу, я проанализировал рынок труда и получил опыт в оформлении работ согласно стандартам.



\newpage
\begin{thebibliography}{99}
\bibitem{bib1} 1.	Зорич В. А. Математический анализ – Часть I – Москва 2019 – 306с.

\bibitem{bib2} 2.	HeadHunter – URL: \url{https://spb.hh.ru} (дата обращения: 27.09.2022).
\end{thebibliography}







\end{document}
