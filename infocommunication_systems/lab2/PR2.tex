\documentclass[14pt]{extreport}
\usepackage{gost}
\usepackage{hyperref}
\usepackage{makecell}
\usepackage{ragged2e}
\justifying

\makeatletter
\@addtoreset{figure}{part}% Reset figure numbering at every part
\makeatother
\renewcommand{\thefigure}{\arabic{figure}}% Figure number is part.figure
\renewcommand{\thetable}{\arabic{table}}



%Тут можно вставить дополнительные пакеты

\begin{document}
\pagestyle{empty} %  выключаем нумерацию
\includepdf[pages=-,pagecommand={}]{title_page.pdf}

\pagestyle{plain} % включаем нумерацию
\tableofcontents
 



\intro\label{intro}



Целью данное практической работы является построение таблицы, отражающей мои карьерные ожидания. Данная практическая работа направлена на исследование и анализ рынка труда, анализ образовательной программы, составление, создание страницы с математическим текстом и формулами, а также оформление всего документа согласно ГОСТ 7.32.


\chapter{МАТЕМАТИЧЕСКИЙ ТЕКСТ\label{chapter1}}
\section{Пример оформления математического текста}

Поскольку обе части этой формулы одновременно меняют знак при перестановке a и b, то формула справедлива при любом соотношении величин a и b,
т. е. как при a $\leq b$, так и при $a \geq b$.

На упражнениях по анализу формула Ньютона — Лейбница большей ча-
стью используется только для вычисления стоящего слева интеграла, и это
может породить несколько искаженное представление об ее использовании.
На самом деле положение вещей таково, что конкретные интегралы редко
находят через первообразную, а чаще прибегают к прямому счету на ЭВМ
с помощью хорошо разработанных численных методов. Формула Ньютона—
Лейбница занимает ключевую, связывающую интегрирование и дифферен-
цирование, позицию в самой теории математического анализа, в которой
она, в частности, получает далеко идущее развитие в виде так называемой
общей формулы Стокса

Примером того, как формула Ньютона— Лейбница используется в самом
анализе, может служить уже материал следующего пункта настоящего пара-
графа.

\subsubsection{3. Интегрирование по частям в определенном интеграле и формула Тейлора}

Утверждение 1. Если функции $u(x)$ и $v(x)$ непрерывно дифференцируемы
на отрезке с концами a и b, то справедливо соотношение \\
\begin{equation}
\int_{a}^{b}  \,(u\cdot v)(x)dx=(u \cdot v)(x)| _a^b - \int_{a}^{b} \, (v \cdot u) (x) dx
\end{equation}

Эту формулу принято записывать в сокращенном виде
\begin{equation}
	\int_{a}^{b}  \,udv = u \cdot v | _a^b - \int_{a}^{b} \, vdu
\end{equation}

и называть формулой интегрирования по частям в определенном интеграле.

$\blacktriangleleft$ По правилу дифференцирования произведения функций имеем
\begin{center}
	$
	(u \cdot v) (x) = (u \cdot v) (x) + (u \cdot v)(x)
	$
\end{center}
По условию все функции в этом равенстве непрерывны, а значит, и инте-
грируемы на отрезке с концами a и b. Используя линейность интеграла и
формулу Ньютона— Лейбница, получаем
\begin{equation}
	(u \cdot v) (x) |^a_b = \int_{a}^{b}(u \cdot v)(x)dx+\int_{a}^{b}(u \cdot v)(x)(dx) \blacktriangleright
\end{equation}

В качестве следствия получим теперь формулу Тейлора с интегральным
остаточным членом
\\

Пусть на отрезке с концами a и x функция $t \Rightarrow  f (t)$ имеет n непрерыв-
ных производных. Используя формулу Ньютона — Лейбница,
проделаем следующую цепочку преобразований, в которых все дифференци-
рования и подстановки производятся по переменной t:

\begin{multline}
	f(x)-f(a)=\int_{a}^{x}dt = - \int_{a}^{x}f(t)(x-t)dt=\\
	= -f(t)(x-t)|^x_a + \int_{a}^{x}f(t)(x-t)dt =\\
	=f(a)(x-a)-\frac{1}{2}\int_{a}^{x}f(t)((x-t)^2)dt=\\
	=f(a)(x-a)-\frac{1}{2}f(t)(x-t)^2|^x_a+\frac{1}{2}\int^x_a f(t)(x-t)^2dt=\\
	=f(a)(x-a)+\frac{1}{2}f(a)(x-a)^2-\frac{1}{2 \cdot 3} \int^x_a f(t) (x-t)^3 dt = \dots \\
	\dots = f(a)(x-a)+\frac{1}{2}f(a)(x-a)^2 + \dots \\
	\dots + \frac{1}{2 \cdot 3 \cdot  \dots \cdot (n-1) } f^(n-1)(a)(x-a)^{n-1}+r_n-1(a;x) \\
\end{multline}
где

\begin{equation}
	r_{n-1}(a;x)=\frac{1}{(n-1)!}\int^x_a f^{(n)}(t)(x-t)^{n-1}dt	
\end{equation}

Итак, доказано следующее

Утверждение 2. Если функция $t \Rightarrow  f (t)$ имеет на отрезке с концами a
и x непрерывные производные до порядка n включительно, то справедлива
формула Тейлора

\begin{center}
	$
	f(x)=f(a)+\frac{1}{1!}f(a)(x-a)+\dots+\frac{1}{(n-1)!}f^{(n-1)}(a)(x-a)^{n-1}+r_{n-1}(a;x)
	$
\end{center}



с остатком $r_{n-1}(a; x)$, представленным в интегральной форме.
Отметим, что функция $(x - t)^{n-1}$ не меняет знак на отрезке с концами a и
x, и поскольку функция $t\Rightarrow  f ^{(n)}(t)$ непрерывна на этом отрезке, то по первой
теореме о среднем на нем найдется такая точка $\mathfrak{Z}$ , что
\begin{multline}
	r_{n-1}(a;x)=\frac{1}{(n-1)!} f^{(n)}(t)(x-t)^{n-1}dt = \frac{1}{(n-1)!} f^(n)(\mathfrak{Z}) \int^x_a (x-t)^{n-1}dt=\\
	=\frac{1}{(n-1)!} f^{(n)}(\mathfrak{Z})(-\frac{1}{n}(x-t)^n)|^x_a=\frac{1}{n!}f^{(n)}(\mathfrak{Z})(x-a)^n
	1
\end{multline}

Мы вновь получили знакомую форму Лагранжа остаточного члена форму-
лы Тейлора. (На основании задачи 2 b) из предыдущего параграфа, можно
считать, что $\mathfrak{Z}$ лежит в интервале с концами $a, x$.)

Это рассуждение можно было бы повторить, вынося из-под знака интеграла $f (n)(\mathfrak{Z})(x - \mathfrak{Z})^{n-k}$ , где $k \in [1, n]$. Значениям $k = 1$ и $k = n$ отвечают
получаемые при этом соответственно формулы Коши и Лагранжа остаточно-
го члена.

\subsubsection{4. Замена переменной в интеграле.} Одной из основных формул ин-
тегрального исчисления является формула замены переменной в опреде-
ленном интеграле. Эта формула в теории интеграла столь же важна, как в
дифференциальном исчислении формула дифференцирования композиции
функций, с которой она может быть при определенных условиях связана
посредством формулы Ньютона— Лейбница

\begin{landscape}
\chapter{ОБЗОР РЫНКА ВАКАНСИЙ\label{chapter2}}

\section{Front-end разработчик}

Таблица 1 - Frontend разработчик
\begin{longtable}[H]{lp{1\linewidth}}
\caption{Frontend разработчик \label{table1}}


\centering

\begin{small}


    \begin{tabular}{|c|p{5,5cm}|p{6cm}|p{5cm}|p{5cm}|}
	\hline 
	\makecell{№ \\ п.п.} &	\makecell{Название должности,\\ ссылка} &	\makecell{Требования} & 	\makecell{Дисциплины \\ из учебного \\плана} &	\makecell{Преимущества и \\недостатки}  \\ 
	\hline 
	1	& Senior Frontend developer

(\url{https://goo.su/fUzWQ})
& •	Знание JavaScript, HTML, CSS

•	Знание Angular, TypeScript

 &	•	Web-программирование
 
•	Программирование
 & + Гибкий график

+ Высокая заработная плата

 \\
	\hline
	2 & Senior Frontend developer (React)
	
(\url{https://goo.su/i9QwFZz})
& •	Знание JavaScript, HTML, CSS

•	Знание React, TypeScript, Redux
&	•	Web-программирование
 
•	Алгоритмы и структуры данных
&+	Официальное трудоустройство 

+ Гибкий график

- Заработная плата не такая большая\\


	\hline
	3 & Middle Frontend разработчик
	
(\url{https://goo.su/taUjVfT}) 
& •	Знание JavaScript, HTML, CSS

•	Знание VueJS, GraphQL
&
•	Информатика

•	Программирование
&
+	Не требуют много опыта

-	Полная занятость
\\
	\hline


    \end{tabular}
\end{small}
\end{longtable}

\addtocounter{table}{-1}
\newpage
Продолжение таблицы 1
\begin{longtable}[H]{lp{1\linewidth}}
\caption{Продолжение таблицы 1}



\centering
\begin{small}


    \begin{tabular}{|c|p{5,5cm}|p{6cm}|p{5cm}|p{5cm}|}
	\hline 
	\makecell{№ \\ п.п.} &	\makecell{Название должности,\\ ссылка} &	\makecell{Требования} & 	\makecell{Дисциплины \\ из учебного \\плана} &	\makecell{Преимущества и \\недостатки}  \\ 
	\hline 
4 & Frontend разработчик

(\url{https://goo.su/N1SU3Vi}) 
 & •	Знание JavaScript, HTML, CSS 

•	Знание React, Redux, GrapQL &
•	Алгоритмы и структуры данных &
+	Комфортный офис

+	Гибкий график

-	Полная занятость\\
\hline
5 & Frontend разработчик 

(\url{https://goo.su/KdSg}) & •	Знание JavaScript, React, Redux & •	Web-программирование & 
+	Офис в центре Петербурга

-	Нужен опыт работы\\

	\hline

    \end{tabular}
    \end{small}
\end{longtable}

Вывод: данная профессия имеет высокий спрос на рынке труда, хорошо оплачивается. Меньший порог входа и большое количество вакансий с частичной или проектной занятостью позволяет совмещать работу и учебу. Освоив дисциплины из моей образовательной программы, я, вероятно смогу получить эту профессию.




\newpage
\section{Back-end разработчик}
Таблица 2 – Backend разработчик
\begin{longtable}[H]{lp{1\linewidth}}

\caption{Backend разработчик}


\centering

\begin{small}


    \begin{tabular}{|c|p{5,5cm}|p{6cm}|p{5cm}|p{5cm}|}
	\hline 
	\makecell{№ \\ п.п.} &	\makecell{Название должности,\\ ссылка} &	\makecell{Требования} & 	\makecell{Дисциплины \\ из учебного \\плана} &	\makecell{Преимущества и \\недостатки}  \\ 
	\hline 
	1	& Senior Java / Kotlin developer
	
(\url{https://goo.su/hzrLf}) &
•	Знание Java, Kotlin

•	Понимание модели OSI, HTTP &
•	Прикладное программирование

•	Разработка приложений на Java &
+	Высокая заработная плата


+	Комфортный офис

 \\

	\hline
	2	& Java Developer
	
(\url{https://goo.su/UZgR}) &
•	Знание Java, Java SE, Spring

•	Опыт работы с SQL базами данных

&

•	Прикладное программирование

•	Разработка приложений на Java &
+	Гибкий график


-	Маленькая заработная плата\\
	\hline 
	3	& Java backend developer
	
(\url{https://goo.su/yjWIc}) &
•	Знание Java, Kotlin

•	Знание Java Spring &

•	Разработка приложений на Java &
+	Возможность удаленной работы

+	Гибкий график
 \\


	\hline


    \end{tabular}
    \end{small}
\end{longtable}




\addtocounter{table}{-1}
\newpage
Продолжение таблицы 2
\begin{longtable}[H]{lp{1\linewidth}}
\caption{Продолжение таблицы 2}

\centering

\begin{small}


    \begin{tabular}{|c|p{5,5cm}|p{6cm}|p{5cm}|p{5cm}|}
	\hline 
	\makecell{№ \\ п.п.} &	\makecell{Название должности,\\ ссылка} &	\makecell{Требования} & 	\makecell{Дисциплины \\ из учебного \\плана} &	\makecell{Преимущества и \\недостатки}  \\ 
	\hline 
	4	& Java/Kotlin developer
	
(\url{https://goo.su/QqoRDVJi}) &
•	Знание Java Spring SE, Kotlin 
•	Опыт работы с SQL базами данных 
& 
•	Разработка приложений на Java &
+	Высокая заработная плата

+	Удаленная работа

 \\

	\hline
5	& Senior Java Developer

(\url{https://goo.su/lsU1F}) &
•	Знание Java, Java SE, Spring

•	Понимание ООП, модели OSI
 &
•	Разработка приложений на Java

•	Алгоритмы и структуры данных &
+	Гибкий график

+	Возможность удаленной работы

-	Маленькая заработная плата \\


	\hline 


    \end{tabular}
    \end{small}
\end{longtable}
Вывод: данная профессия имеет меньший спрос на рынке труда, относительно предыдущей профессии, но как правило имеет большую заработную плату. На рынке труда меньше вакансий с проектной и частичной занятостью.  Освоив дисциплины из моей образовательной программы, я, вероятно смогу получить эту профессию.







\newpage
\section{DevOps инженер}
Таблица 3 – DevOps инженер
\begin{longtable}[H]{lp{1\linewidth}}
\caption{DevOps инженер \label{table3}}


\centering

\begin{small}


    \begin{tabular}{|c|p{5,5cm}|p{6cm}|p{5cm}|p{5cm}|}
	\hline 
	\makecell{№ \\ п.п.} &	\makecell{Название должности,\\ ссылка} &	\makecell{Требования} & 	\makecell{Дисциплины \\ из учебного \\плана} &	\makecell{Преимущества и \\недостатки}  \\ 
	\hline 
	1	& DevOps инженер
	
(\url{https://goo.su/wJxfQ}) &
•	Опыт работы с Linux, Docker

• Знание Python, SQL &
•	Информационная безопасность

•	Программирование &
+	Высокая заработная плата

+	Комфортный офис
 \\


	\hline
	2	& DevOps Engineer (Kubernetes)
	
(\url{https://goo.su/RnjXO}) &
•	Понимание архитектуры и принципов разработки web-приложений

•	Опыт работы с Linux, Kubernetes, SQL &
•	Информатика

•	Информационная безопасность &
+	Удаленная работа

-	Полная занятость
\\

	\hline 
	3	& DevOps Engineer (GCP/Jenkins)
	
(\url{https://goo.su/EM9s}) 	&
•	Знание Golang

•	Умение работать с Linux, Docker, k8 &
•	Информационная безопасность

•	Информатика &
+	Высокая заработная плата

+	

-	Маленькая кампания \\


	\hline


    \end{tabular}
    \end{small}
\end{longtable}

\newpage
Продолжение таблицы 3
\begin{longtable}[H]{lp{1\linewidth}}
\caption{Продолжение таблицы 3}

\centering

\begin{small}


    \begin{tabular}{|c|p{5,5cm}|p{6cm}|p{5cm}|p{5cm}|}
	\hline 
	\makecell{№ \\ п.п.} &	\makecell{Название должности,\\  ссылка} &	\makecell{Требования} & 	\makecell{Дисциплины \\ из учебного \\плана} &	\makecell{Преимущества и \\недостатки}  \\ 
	\hline 
	4	& DevOps инженер SRE 
	
(\url{https://goo.su/zB5JX}) &
•	Знание Golang, Python

•	Опыт работы с Linux, Docker, AWS &
•	Информационная безопасность

•	Информатика &
+	Высокая заработная плата

-	Офис далеко от метро
\\



	\hline
	5	& DevOps инженер 
	
(\url{https://goo.su/yuGzpG}) &
•	Знание Python

•	Умение работать с Linux, Docker
	&
•	Информатика

• Информационная безопасность & 
+	Гибкий график
 
+	Высокая заработная плата 
\\

	\hline 


    \end{tabular}
    \end{small}
\end{longtable}

Вывод: данная профессия набирает популярность на рынке труда в России. На рынке присутствует не так много вакансий, но все они очень хорошо оплачиваются. Нужно хорошо разбираться в нескольких областях сразу. Освоив дисциплины из моей образовательной программы, я, вероятно смогу получить эту профессию.
\end{landscape}



\conclusions

Практическая работа выполнена. Построена таблица желаемых должностей и оформлен математический текст согласно ГОСТ 7.32. Выполняя работу, я проанализировал рынок труда и получил опыт в оформлении работ согласно стандартам.



\newpage
\begin{thebibliography}{99}
\bibitem{bib1} 1.	Зорич В. А. Математический анализ – Часть I – Москва 2019 – 306с.

\bibitem{bib2} 2.	HeadHunter – URL: \url{https://spb.hh.ru} (дата обращения: 27.09.2022).
\end{thebibliography}







\end{document}
